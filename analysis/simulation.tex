\section{Simulation}\label{sec:analysis.simulation}
There are certain kinematic regions of \abbr{CLAS} in which physics events are not being recorded properly i.e.~the area dividing each sector in \abbr{CLAS}. Furthermore each sector in \abbr{CLAS} is asymmetric in the acceptance of events due to subsystem inefficiencies such as inoperable \abbr{DC} wires, \abbr{PMT} inefficiencies, dead scintillator strips the the \abbr{TOF} and \abbr{ST} subsystems. When a triggered event is recorded and reconstructed these asymmetric inefficiencies factors are reflected and must be carefully understood because these factors are properties of the \abbr{CLAS} detector and independent of any physics that occurred. To properly understand the detector effects on the data, \abbr{CLAS} utilizes a \abbr{GEANT} simulation package know as \abbr{GSIM}. To prepare an event for \abbr{GSIM} the program \abbr{GAMP2PART} converts a text file, containing the 4-momentum of the generated event, into a suitable file format for \abbr{GSIM}. \abbr{GSIM} then simulates the passage of these particles through the \abbr{CLAS} detector and generates the associated \abbr{ADC} and \abbr{TDC} information from detector hits. \abbr{GSIM} takes into account detector inefficiencies described in the \abbr{\texttt{CLAS\_CALDB\_RUNINDEX}}. The \abbr{CLAS\_CALDB\_RUNINDEX} is an array of information about each subsystem's inefficiency that was derived during the \g12 calibration process. The \abbr{GSIM} simulated hits are then ``post-processed'' by smearing the \abbr{TDC} and \abbr{ADC} hits to imitate the observed resolution of the detector subsystems using the program \abbr{GPP} (\abbr{GSIM} post-processor). \abbr{GPP} also removes detector hits due to inefficient \abbr{DC} wires. The simulation output processed with \abbr{GPP} is then reconstructed with \texttt{a1c}, the same program used to reconstruct data events. The reconstructed simulation is subject to the same scrutiny as real data events, undergoing all the cuts (Sec.~\ref{sec:analysis.data.reduction}), corrections (Sec.~\ref{sec:analysis.corrections}), and kinematic fitting (Sec.~\ref{sec:analysis.fitting}), as the real data except for beam corrections (Sec.~\ref{sec:analysis.corrections.beam}).

\subsection{Simulation Verification}\label{sec:analysis.accept.verify}
Part of understanding the simulation output is understanding how well the simulation mimics the real data. To investigate this, 26000 real \epem events were treated as generated events and inputted into the \abbr{GAM2PART}$\to$\abbr{GSIM}$\to$\abbr{GPP}$\to$\texttt{a1c} chain. Of the 26000 inputted, only 100 were successfully reconstructed through the simulation chain. The source of this low efficiency was due to the calibrations entries for the \abbr{CC} and \abbr{EC} in the \abbr{CLAS\_CALDB\_RUNINDEX} not having values in which would set the ``PEDESTAL'' values appropriately for simulation. The calibrations constants in the \abbr{CLAS\_CALDB\_RUNINDEX} were correct for data reconstruction, but not for simulation reconstruction of \epem in the \abbr{CC} and \abbr{EC} subsystems. It was also discovered that the \abbr{CC} and \abbr{EC} subsystems should be simulated with ``RUN 10'' constants instead of the normal ``RUN 56855'' used by the \g12 group. ``RUN 56855'' is a special run benchmarked to have the best calibrations and required to properly simulate the \abbr{ST}, \abbr{DC}, and \abbr{TOF} subsystems. To rectify this, a special \abbr{CLAS\_CALDB\_RUNINDEX} was created, changing ``RUN 10'' for to have ``RUN 56855'' constants for all subsystems except the \abbr{CC} and \abbr{EC} subsystems which were kept at ``RUN 10'' constants. Inputting the 26000 real \epem events into the simulation chain using the \abbr{CLAS\_CALDB\_RUNINDEX} \emph{RunIndexg12\_leptons\_and\_photons} outputted $\approx$~24700 \epem reconstructed events, a $\approx$~95~\% efficiency.

The missing 5~\% was a result of ``time-based'' and ``hit-based'' tracking failures as briefly mentioned in Sec.~\ref{sec:data.cook}. The events that failed ``hit-based'' tracking contributes a 3.75~\% overall event inefficiency. The cause of the ``hit-based'' failure was never determined, but it was thought to have also occur in the cooking of the data. Therefore since it did occur in the data reconstruction this was considered to cancel the inefficiency of the simulation.

The ``time-based'' failure was due to a random bug in the processing of the \abbr{TDC} element information of \abbr{ST} (\abbr{STN0}) and the \abbr{ADC} element information of \abbr{ST} (\abbr{STN1}) raw data banks. The bug miscalculated the tracks sector exiting the \abbr{ST} even as the hit element of the \abbr{ST} matched that to the track in the \abbr{DC}. If the track failed due to this error, it usually passed ``time-based'' on the second or third pass of the ``time-based'' tracking if another particle passed ``time-based'' during the initial pass. The probability that a track failed initial ``time-based" tracking was $\approx$~.23\%. The probability that this failed event would pass ``time-based" tracking after another pass was $\approx99.78\%$. The average inefficiency for three charged track events for data was 0.0125\%

%This inefficiency depends on the distance the track is from the z-vertex position and can be seen in Fig.~\ref{fig:classt.ineffII}.
%\begin{figure}[h!]\begin{center}
%\includegraphics[width=0.8\figwidth,height=0.7\qfigheight]{\grpath/hall-b/st_issue_4_thesis.pdf}
%\caption[Start Counter Inefficiency]{\label{fig:classt.ineffII}{\coloronline}Plot showing the inefficiency of the start counter from data events, red-solid line is the inefficiency of reconstruction based solely on hit-based tracking, blue-dashed line is inefficiency of start counter, black-solid is combined. }
%\end{center}\end{figure}

\FloatBarrier
\input{analysis/simtrigger}
\section{PLUTO++ Event Generator}\label{sec:pluto}

Pluto~\cite{PLUTO} is a Monte-Carlo event generator designed for the study of hadronic interactions and heavy ion reactions in \abbr{HADES}, \abbr{FAIR} and upcoming \abbr{PANDA} collaborations. The versatility of Pluto enables its use as an event generator for photoproduction in \abbr{CLAS}. For hadronic interactions, Pluto can generate interactions from pion production threshold to intermediate energies of a few~GeV per nucleon. The entire software package is based on ROOT and uses ROOT's embedded C++ interpreter to control the generation of events. Programming event reaction can be set up with a few lines of ROOT macro code without detailed knowledge of programming. Some features in Pluto are, but not limited to;
\begin{itemize}
\item Ability to generate events in phase space.
\item Ability to generate events with a continuous bremsstrahlung photon beam.
\item Ability to generate events weighted by a user defined $t$-slope.
\item Ability to generate events weighted by a user defined cross-section.
\begin{itemize}
\item Total cross section can be inputted via functional form or histogram.
\item Differential cross sections can be inputted via functional forms or histograms for specific beam energies up to 110 histograms relating to intervals of beam energy.
\end{itemize}
\item Ability to generate events that decay via already established physics parameters, i.e.~transition form factors.
\item Ability to generate events that decay via modified established physics parameters.
\item Ability to generate events with multiple production channels, weighted by user inputted cross-section probability.
\item Ability to generate events with multiple decay channels, weighted by user inputted branching ratio.
\item Ability to perform vertex smearing.
\item Ability to create virtual detectors.
\end{itemize}

For the analysis presented in this work, Pluto was used in conjunction with known differential cross sections to verify simulation momentum smearing and tagger resolution, Sec.~\ref{sec:analysis.simsmear.verify}. Pluto was also utilized as a phase space generator in this analysis, to perform a ``tune'' on the kinematic fitter, Sec.~\ref{sec:analysis.fitting}, to calculate the acceptance corrections Sec.~\ref{sec:results.acceptance}, and to calculate the normalization Sec.~\ref{sec:results.normalization}.
\subsection{Simulation Verification}\label{sec:analysis.accept.verify}
Part of understanding the simulation output is understanding how well the simulation mimics the real data. To investigate this, 26000 real \epem events were treated as generated events and inputted into the \abbr{GAM2PART}$\to$\abbr{GSIM}$\to$\abbr{GPP}$\to$\texttt{a1c} chain. Of the 26000 inputted, only 100 were successfully reconstructed through the simulation chain. The source of this low efficiency was due to the calibrations entries for the \abbr{CC} and \abbr{EC} in the \abbr{CLAS\_CALDB\_RUNINDEX} not having values in which would set the ``PEDESTAL'' values appropriately for simulation. The calibrations constants in the \abbr{CLAS\_CALDB\_RUNINDEX} were correct for data reconstruction, but not for simulation reconstruction of \epem in the \abbr{CC} and \abbr{EC} subsystems. It was also discovered that the \abbr{CC} and \abbr{EC} subsystems should be simulated with ``RUN 10'' constants instead of the normal ``RUN 56855'' used by the \g12 group. ``RUN 56855'' is a special run benchmarked to have the best calibrations and required to properly simulate the \abbr{ST}, \abbr{DC}, and \abbr{TOF} subsystems. To rectify this, a special \abbr{CLAS\_CALDB\_RUNINDEX} was created, changing ``RUN 10'' for to have ``RUN 56855'' constants for all subsystems except the \abbr{CC} and \abbr{EC} subsystems which were kept at ``RUN 10'' constants. Inputting the 26000 real \epem events into the simulation chain using the \abbr{CLAS\_CALDB\_RUNINDEX} \emph{RunIndexg12\_leptons\_and\_photons} outputted $\approx$~24700 \epem reconstructed events, a $\approx$~95~\% efficiency.

The missing 5~\% was a result of ``time-based'' and ``hit-based'' tracking failures as briefly mentioned in Sec.~\ref{sec:data.cook}. The events that failed ``hit-based'' tracking contributes a 3.75~\% overall event inefficiency. The cause of the ``hit-based'' failure was never determined, but it was thought to have also occur in the cooking of the data. Therefore since it did occur in the data reconstruction this was considered to cancel the inefficiency of the simulation.

The ``time-based'' failure was due to a random bug in the processing of the \abbr{TDC} element information of \abbr{ST} (\abbr{STN0}) and the \abbr{ADC} element information of \abbr{ST} (\abbr{STN1}) raw data banks. The bug miscalculated the tracks sector exiting the \abbr{ST} even as the hit element of the \abbr{ST} matched that to the track in the \abbr{DC}. If the track failed due to this error, it usually passed ``time-based'' on the second or third pass of the ``time-based'' tracking if another particle passed ``time-based'' during the initial pass. The probability that a track failed initial ``time-based" tracking was $\approx$~.23\%. The probability that this failed event would pass ``time-based" tracking after another pass was $\approx99.78\%$. The average inefficiency for three charged track events for data was 0.0125\%

%This inefficiency depends on the distance the track is from the z-vertex position and can be seen in Fig.~\ref{fig:classt.ineffII}.
%\begin{figure}[h!]\begin{center}
%\includegraphics[width=0.8\figwidth,height=0.7\qfigheight]{\grpath/hall-b/st_issue_4_thesis.pdf}
%\caption[Start Counter Inefficiency]{\label{fig:classt.ineffII}{\coloronline}Plot showing the inefficiency of the start counter from data events, red-solid line is the inefficiency of reconstruction based solely on hit-based tracking, blue-dashed line is inefficiency of start counter, black-solid is combined. }
%\end{center}\end{figure}

\FloatBarrier
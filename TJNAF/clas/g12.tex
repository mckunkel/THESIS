\section{\g12 Running Conditions}\label{sec:clas.g12}

Three \abbr{CLAS} analysis proposals (\abbr{04-005}\cite{clas.proposal.hyclas}, \abbr{04-017}\cite{clas.proposal.superg} and \abbr{08-003}\cite{clas.proposal.pion}) defined the experimental and theoretical basis for the \g12 running period. \label{sec:clas.hyclas}The \abbr{04-005} experiment, \emph{Search for New Forms of Hadronic Matter in Photoproduction}, also called \abbr{HyCLAS}, had a meson spectroscopy focus with multiple charged particle final states such as
\begin{eqnarray}
    \gamma \p & \rightarrow & \p \pi^+ \pi^- \pi^0, \\
    \gamma \p & \rightarrow & \n \pi^+ \pi^+ \pi^-, \\
    \gamma \p & \rightarrow & \p \Kp \Km \eta, \\
    \gamma \p & \rightarrow & \n \Kp \Kp \pi^-, \\
    \gamma \p & \rightarrow & \Delta^{++} \eta \pi^-, \\
    \gamma \p & \rightarrow & \p \p \bar{\p}.
\end{eqnarray}
The physics involved with \abbr{HyCLAS} required the configuration of \abbr{CLAS} to provide the largest acceptance for these multiple particle final states. Phase-space generated events of $\mathrm{\gamma p \rightarrow p \pi^+ \pi^- \pi^0}$ were simulated (see page~30 of \cite{clas.proposal.hyclas}) with the $t$-slope obtained from the \textit{g6c} experiment. The primary requirement for the greatest acceptance of such events was to have the target up-stream (see Sec.~\ref{sec:clas.tgt}) of the normal position at the ``center'' of \abbr{CLAS}. This target placement gave better acceptance for particles close to the beam-line but sacrificed large momentum-transfer events where the final state particles were more than about 70$^\circ$ away from the beam-line.

\label{sec:clas.superg}The \abbr{04-017} experiment, \emph{Study of Pentaquark States in Photoproduction off Protons}, also called \abbr{Super-G}, was founded on a search for the $\Theta^+$ and $\Xi^{--}_{5}$, so-called \emph{penta-quarks}, as well as a study of the ``conventional'' $\Xi$ spectrum (see page~16 of \cite{clas.proposal.superg}.) This analysis is part of the latter topic. The running requirements were similar to that of \abbr{HyCLAS} with the need for a higher energy beam. An examination of the ground state $\Xi^-$ reaction:
\[
    \mathrm{\gamma p \rightarrow \Xi^- K^+ K^+},
\]
provides a starting point for this analysis. The threshold energy of the incident photon ($E_\gamma$) is given by
\begin{equation}
    E_{\gamma} = \frac{m_{\Xi}^2 + 4 m_{\mathrm{K}}^2 + 4 m_{\rule{0pt}{1.4ex}\Xi} m_{\rule{0pt}{1.4ex}\mathrm{K}} - m_{\p}^2}{2 m_{\p}},
\label{eqn:xi.threshold}
\end{equation}
where $m_{\rule{0px}{1.4ex}\Xi}$ is the mass of the $\Xi$, $m_{\rule{0px}{1.4ex}\mathrm{K}}$ is the mass of the K$^{+}$, and $m_{\p}$ is the proton (target particle) mass. For the ground state $\Xi(1320)$ which has a mass of 1.322~GeV, the threshold energy $E_\gamma$ is 2.4 GeV. Since the beam
(photon) and the target (proton) are both known quantities, we can measure the two kaons and calculate the $\Xi^-$ through ``missing mass'' which is discussed in detail in Chapter~\ref{sec:analysis}. There is a minimum transverse momentum the final state particles must have to be measured by \abbr{CLAS}, otherwise they would travel right down the beam line. Therefore, in order to detect the two kaons with \abbr{CLAS}, a photon energy approximately 0.5~GeV above threshold is required. This corresponds to 2.9~GeV in the reaction for the ground state $\Xi(1320)$.

The third proposal, \abbr{08-003}, titled \emph{The $\gamma \p \rightarrow \pi^+ \n$ Single Charged Pion Photoproduction}, was approved just before the \g12 run period started. This was added onto \g12 as part of the physics to be done with the data collected. It required a single track trigger (see Sec.~\ref{sec:data.trig} on page~\pageref{sec:data.trig}) and lower current. This configuration allowed the data from these special runs to be included in analyses of the ``production'' \g12 data set.

At the beginning of \g12 run, approval came for purchasing gas for the \v{C}erenkov subsystem of \abbr{CLAS}; see Sec.~\ref{sec:clas.cc}. The \v{C}erenkov counters were filled and turned on two weeks into the running period enabling the separation of electons from pions. As a result, a whole new set of leptonic physics became available in what was already a very rich data set.

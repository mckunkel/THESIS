\subsection{SAID}\label{sec:intro.said}

SAID~\cite{SAID} is a repository of experimental data and an interactive analysis facility, allowing to compare and extract data and partial wave solutions (PWA) for a variety of photoproduction, electro-production and pion production reactions. It was created by R. A. Arndt and L. D. Roper for the use of verifying model calculations against measured/fitted data, compare model calculations against SAID predictions for unmeasured observables, experimental planning, and simulations and event generators.

SAID is based upon the theoretical framework given in~\ref{sec:isospin}. SAID generates resonance couplings, in terms of angular momentum and isospin quantum numbers, that are extracted from a fit-based determination of multipoles using both an energy-dependent and an energy-independent parametrization. The photoproduction amplitude is assumed to be in the form of a Breit-Wigner and a background term in the form of~\cite{ar90};
\begin{align}
A=A_l(1+iT_\pi)=A_r\left(\frac{k_0q_0}{kq}\right)^{\frac{1}{2}} \frac{W_0\sqrt{\Gamma \Gamma_{\gamma}}}{W_0^2-W^2-iW_0\Gamma} \ ,
\end{align}
in which $A_l$ is the background parameter, $W_0$, $\Gamma$ and $\Gamma_{\gamma}$ are functions of the full width $\Gamma_{0}$ and $A_R$ being the resonant parameter in the form of;
\begin{align}
A_r=\frac{\mu}{q}\left(\frac{k}{q}\right)^l\sum_{n=0}^{N}p_n\left(\frac{E_{\pi}}{\mu}\right)^n
\end{align}
where $k_0$ and $q_0$ are the pion and photon momenta at the resonance energy, $\mu$ is the pion mass, $E_{\pi}$ is the pion kinetic energy in the lab frame and $p_n$ is a free parameter. The background term is expanded as a set of Legendre polynomial terms with associated free parameters along with a sum of a pseudoscalar Born partial waves, which are determined by fitting the data. Multipoles can then be extracted by a fit of $A$ close to the resonance position.

\subsection{Summary}
A full set of differential cross-sections and polarization observables is required for the determination of multipoles. These can be related to the invariant amplitudes as functions of $W$ and $\cos\theta$. For the separation of the different isospin contributions, both the proton and the neutron pion-photoproduction measurements are needed. An understanding of the invariant amplitudes, and subsequent CGLN structure functions, can provide information on the multipole transitions taking place. This  allows to get direct information about the quantum numbers of the produced resonant states and constrain their position, widths and couplings.
The work of this thesis is devoted to measuring of the differential cross-sections for better fit determinations and possible resonance searches using the SAID parameterizations.
\section{Systematic Uncertainty}\label{sec:results.systematics}

In this section, the calculations leading up to the systematic of the measurement will be discussed. Systematic errors are caused the controls of the experiment, such as flux, simulation, density and length of the $\ell H_2$ target and also systematic errors are caused by various analytical tools used, such as the kinematic fitter.
\subsection{Branching Ratio Systematic Uncertainty}
The branching ratios for the two topologies used to measure the cross-section were obtained from \abbr{PDG}\label{abbr:pdg}~\cite{pdg} and are listed again in Table~\ref{tab:brspecs} with their associated errors. Uncorrelated quantities that are summed as,
\begin{align}
f = \sum_{i = 1}^{M}a_iP_i  
\end{align}
have errors as
\begin{align}
\sigma_f = \sqrt{\sum_{i = 1}^{M}\left(a_i\sigma_i\right)^2}.  
\end{align}
Therefore
\begin{align}
 \frac{\Gamma}{\Gamma_{tot}} &  = \frac{\Gamma_{\pi^{0}\rightarrow e^{+}e^{-}\gamma}}{\Gamma_{tot}} + \frac{\Gamma_{\pi^{0}\rightarrow \gamma \gamma \to e^{+}e^{-}\gamma}}{\Gamma_{tot}}  \\ & = \frac{\Gamma_{\pi^{0}\rightarrow e^{+}e^{-}\gamma}}{\Gamma_{tot}} + \frac{\Gamma_{\pi^{0}\rightarrow \gamma \gamma}P(\gamma \to  e^{+}e^{-})}{\Gamma_{tot}} \ ,
\end{align}
where $P(\gamma \to  e^{+}e^{-})$ is the probability of photon conversion into $e^+e^-$. To measure $P(\gamma \to  e^{+}e^{-})$, the acceptance for conversion ($P(\gamma \to  e^{+}e^{-})\cdot\eta_{e^+e^-}$) is divided by the acceptance for Dalitz ($\eta_{e^+e^-}$). Fig.~\ref{fig:convprob_all} shows that the conversion probability depends on incident photon energy. A maximum probability of 8\%  per-photon was measured, shown in Fig~\ref{fig:convprob_II} top left plot. 
\begin{figure}[h!]\begin{center}
\subfloat[Probability of Photon Conversion vs. $\cos\theta$][]{ %Feynman diagram of \piz two photon decay
\includegraphics[width=\figwidth,height=\qfigheight]{\grpath/analysis/BRANCHING_RATIO/Converion_Probability_fix.pdf}\label{fig:convprob_I}
}\\
\subfloat[Probability of Photon Conversion vs. $\cos\theta$][]{ %Feynman diagram of \piz Dalitz decay
\includegraphics[width=\figwidth,height=\qfigheight]{\grpath/analysis/BRANCHING_RATIO/Converion_Probability_II_fix.pdf}\label{fig:convprob_II}
}
\caption[Probability of Photon Conversion vs. $\cos\theta$ for various values of $E_\gamma$]{\label{fig:convprob_all}Probability of Photon Conversion vs. $\cos\theta$ for various values of $E_\gamma$. The maximum probability for this analysis was measured in the top left plot of b.}
\end{center}\end{figure}
Therefore
\begin{align}
 \frac{\Gamma}{\Gamma_{tot}} = \frac{\Gamma_{\pi^{0}\rightarrow e^{+}e^{-}\gamma}}{\Gamma_{tot}} + \frac{\Gamma_{\pi^{0}\rightarrow \gamma \gamma}P(\gamma \to  e^{+}e^{-})}{\Gamma_{tot}} = 0.09 \ ,
\end{align}
and has error
\begin{align}
\sigma_f = \sqrt{\left(\frac{1}{\Gamma_{tot}}\right)^2(\sigma^2_{\pi^{0}\rightarrow e^{+}e^{-}\gamma} + \sigma^2_{\pi^{0}\rightarrow \gamma \gamma})  } = 0.0037.  
\end{align}
The energy and $\cos \theta$ dependence of the conversion is accounted for in the acceptance, which is $E_\gamma$ and $\cos \theta$ bin-dependent. 
\begin{table}[h!]
\begin{minipage}{\textwidth}
\begin{center}
\begin{singlespacing}

\caption[Branching Ratios of \piz]{\label{tab:brspecs}Branching ratio and errors used in $\frac{d\sigma}{d\cos\theta^{\pi^0}_{C.M.} d\phi}$ measurements \vspace{0.75mm}}

\begin{tabular}{c|c|c}

\hline
Quantity & Value & Error \\
\hline
$\Gamma_{\pi^{0}\rightarrow \gamma \gamma }$& 0.98823 & 0.00034 \\
$\Gamma_{\pi^{0}\rightarrow e^{+}e^{-}\gamma}$ & 0.01174 & 0.00035 \\
$\frac{\Gamma}{\Gamma_{tot}}$ & 0.13 & 0.0037 \\
\hline \hline
\end{tabular}

\end{singlespacing}
\end{center}
\end{minipage}
\end{table}
\vspace{20pt}

\FloatBarrier

\subsection{Cut Based Systematic Uncertainty}
The procedure to determine the systematic uncertainty of the cuts placed on the various kinematic fits was first to calculate an acceptance with a different cut, then to calculate a new total cross-section measurement applying the different cut to the data. The total cross-section was computed at various photon beam energies. Lets denote the original measured total cross-section as $\Xi_1$ and the new total cross-section determined by the new cut as $\Xi_n$, then the systematic error was calculated as.

\begin{align}
\sigma_{cut} = \frac{\left| \Xi_1 - \Xi_n \right|}{\Xi_1}
\end{align}

Some systematic uncertainty depended on the photon energy. All cut based systematics were performed individually, meaning when a cut was changed, the remaining cuts retained their original value, see Table~\ref{tab:cutsystematics} for the values of the cuts that were changed to calculate the systematic error.
%This new cross-section was then compared to the cross-section using the base cuts to determine if a dependence on $\cos\theta^{\pi^0}_{C.M.})$ was present. All systematics of the cut based did not show a $\cos\theta^{\pi^0}_{C.M.})$ dependency for specific beam energies.
\begin{table}[h!]
\begin{minipage}{\textwidth}
\begin{center}
\begin{singlespacing}

\caption[Variance of Data Cut Systematics]{\label{tab:cutsystematics}Different Cuts to analyze systematics \vspace{0.75mm}}

\begin{tabular}{c|c|c|c}

\hline
Cut & Original & Adjusted & Uncertainty \\
\hline
2-C Fit Pull Probability  & 1\% & 10\% & $0.0219$\\
1-C Fit Pull Probability & 1\% & 10\% & $ 0.00216 + 0.01083E_{\gamma}$ \\
4-C Fit Pull Probability  & 1\% & 10\% & $0.00031$\\
Missing Energy Cut  & 75~MeV & 100~MeV & $0.02781$\\
\hline \hline
\end{tabular}

\end{singlespacing}
\end{center}
\end{minipage}
\end{table}
\vspace{20pt}

\begin{figure}[h!]\begin{center}
\includegraphics[width=1.2 \figwidth,height=\hfigheight]{\figures/analysis/All_Cut_Systematic.pdf}
\caption[Plot showing the contribution of the data cut systematic error and the incoming beam dependence of the error]{\label{fig:sys_cut_error} Plot showing the contribution of the data cut systematic error and the incoming beam dependence of the error.}
\end{center}\end{figure}
\FloatBarrier
\subsection{Photon Flux Systematic Uncertainty}
The photon flux calculation should be consistent throughout the experiment. If the flux measurement is not consistent due to corrections made with the live-time, beam corrections or fractional difference in the reported current to the actual current during the photon flux normalization run then a systematic uncertainty would be produced. To study this effect we divided the g12 run period into four groups. The change between group 2 and group 3 was at the run in which the tagger hysteresis was not present, see Sec~\ref{sec:analysis.corrections.beam}. Table~\ref{tab:flux_sys} lists the run groups used for this study.
\begin{table}[h!]
\begin{minipage}{\textwidth}
\begin{center}
\begin{singlespacing}

\caption[Run Groups Used to Determine Photon Flux Systematic Error ]{\label{tab:flux_sys}List of run groups used to determine photon flux systematic error. \vspace{0.75mm}}

\begin{tabular}{c|c|c}

\hline
Run Group & Range & Total Runs \\
\hline
1 & 56605-56798 & 116 \\
2 & 56799-56980 & 116 \\
3 & 56992-57173 & 116 \\
4 & 57174-57317 & 115 \\
\hline \hline
\end{tabular}

\end{singlespacing}
\end{center}
\end{minipage}
\end{table}
\vspace{20pt}

The procedure to determine the systematic error, $\sigma$, of the flux is to calculate the accepted and flux corrected yield, $\Upsilon^c$, for each run group and compare $\Upsilon^c$ to the average accepted and flux corrected yield of all 4 run groups, $\mu^c$. After $\sigma$ is calculated, it was normalized to $N \mu^c$ as to represent the error as a percentage, which later is added in quadrature and  multiplied by the measured cross section to determine the appropriate error. 
\begin{align}
\sigma_{group} = \sqrt{\sum_{i=1}^{N = 4}\left(\Upsilon_i^c - \mu^c\right)},
\end{align}
where
\begin{align}
\mu^c = \frac{1}{N}\sum_{i=1}^{N=4}\Upsilon_i^c
\end{align}
\begin{align}
\sigma_{group}^{normalized} = \frac{\sigma_{group}}{N\mu^c}
\end{align}

\begin{figure}[h!]\begin{center}
\includegraphics[width=1.2 \figwidth,height=\hfigheight]{\figures/analysis/Beam_Fluxsystematic_Error.pdf}
\caption[Plot showing the contribution of the flux systematic error and the incoming beam dependence of the error]{\label{fig:sys_flux_error} Plot showing the contribution of the flux systematic error and the incoming beam dependence of the error.}
\end{center}\end{figure}
\FloatBarrier


 
\subsection{Detector Efficiency Systematic Uncertainty}
Each sector in \abbr{CLAS} can be treated as an individual detector, with its own efficiency and resolution. A systematic uncertainty could arise if one or more of the sectors is not simulated properly. The procedure to determine the systematic error, $\sigma$, of the sector is to calculate the accepted corrected yield, $\Upsilon^c$, for each sector and compare $\Upsilon^c$ to the average accepted corrected yield of all 6 sectors, $\mu^c$. After $\sigma$ is calculated, it was normalized to $N \mu^c$ as to represent the error as a percentage, which later is multiplied by the measured cross section to determine the appropriate error. 
\begin{align}
\sigma_{sector} = \sqrt{\sum_{i=1}^{N = 6}\left(\Upsilon_i^c - \mu^c\right)},
\end{align}
where
\begin{align}
\mu^c = \frac{1}{N}\sum_{i=1}^{N=6}\Upsilon_i^c
\end{align}
\begin{align}
\sigma_{sector}^{normalized} = \frac{\sigma_{sector}}{N\mu^c}
\end{align}
This calculation was performed for various bins of incoming beam energy to determine the beam energy dependence (see Fig.~\ref{fig:sys_sec_error}).

\begin{figure}[h!]\begin{center}
\includegraphics[width= \figwidth,height=0.75\hfigheight]{\figures/analysis/Beam_systematic_Error.pdf}
\caption[The sector systematic uncertainty as a function of the incoming photon energy]{\label{fig:sys_sec_error}The sector systematic uncertainty as a function of the incoming photon energy.}
\end{center}\end{figure}
The sector systematic uncertainty is consistent with the extracted sector systematic uncertainty from the g11 data set~\cite{williams}(seeFig.~\ref{fig:sys_sec_error.compare}).
\begin{figure}[h!]\begin{center}
\includegraphics[width= \figwidth,height=0.75\hfigheight]{\figures/analysis/S_systematic_Error.pdf}
\caption[Comparison of sector systematic uncertainty to g11 measurement]{\label{fig:sys_sec_error.compare}Comparison of sector systematic uncertainty to g11 measurement.}
\end{center}\end{figure}
\FloatBarrier

\subsection{$z$-vertex Cut Systematic Uncertainty}
The systematic uncertainty of the $z$-vertex cut was analyzed by varying the initial vertex cut from $-110 \le z \le -70$ to $-109 \le z \le -71$ for both data and \abbr{MC}. Afterward the procedure for determining the systematic was identical to the method used to determine the ``Cut Based Systematic Uncertainty". The systematic uncertainty from varying the $z$ was 0.0041, shown in Fig.~\ref{fig:results.syserr}

\subsection{Target Systematic Uncertainty}
Since the systematic on the density is 0.02\%,see Sec~\ref{sec:analysis.target_density}, the maximum systematic on the target is due to uncertainty in the length on the target which is 40~cm $\pm$ 0.2~cm. A total systematic on the target was assign to be 0.5\%. 

\subsection{Total Systematic Uncertainty}
The total systematic uncertainty along with a list of the individual systematics is presented in this subsection. The calculation of the total systematic error is 
\begin{align}
\sigma^{sys}_{tot} = \sqrt{\sum_{i=1}^{M}\sigma_i^2}
\end{align}
\begin{table}[h!]
\begin{minipage}{\textwidth}
\begin{center}
\begin{singlespacing}

\caption[Systematics]{\label{tab:systematics}Systematic errors used in $\frac{d\sigma}{d\cos\theta^{\pi^0}_{C.M.} d\phi}$ measurements \vspace{0.75mm}}

%\begin{tabular}{c|c}
\begin{tabular}{p{5cm} | p{7cm}}
\hline
Systematic & Error \\
\hline
Sector  & $ 0.0361 + 0.0065E_{\gamma}$ \\
Flux  & $ -0.00051 + 0.00491E_{\gamma}$ \\
Missing Energy Cut  & $0.02781$ \\
2-C Fit Pull Probability & $0.0219$ \\
1-C Fit Pull Probability  & $ 0.00216 + 0.01083E_{\gamma}$ \\
4-C Fit Pull Probability  & $0.00031$ \\ 
Target  & $0.005$ \\
Branching Ratio  & $0.0037$ \\
Fiducial Cut & $0.024$ \\
$z$-vertex Cut & $0.0041$ \\
Total & $\sqrt{0.0032 +0.00051E_{\gamma} +0.000184E_{\gamma}^2}$ \\
\hline \hline
\end{tabular}

\end{singlespacing}
\end{center}
\end{minipage}
\end{table}
\vspace{20pt}
Figure~\ref{fig:results.syserr} is a pictorial version of Table~\ref{tab:systematics}.
\begin{figure}[h!]\begin{center}
\includegraphics[width=1.1 \figwidth,height=\hfigheight]{\figures/analysis/All_Systematics_new.pdf}
\caption[The contribution of all systematic uncertainties]{\label{fig:results.syserr}The contribution of all systematic uncertainties.}
\end{center}\end{figure}
\FloatBarrier
%\clearpage
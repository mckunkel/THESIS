\section{\label{sec:results.xiulimit}Search for Higher Mass Cascades}

For states that were not seen in the data, a variation on the above technique was used to determine the sensitivity of measuring a signal in a certain location --- i.e. a predetermined mass and width. For each energy bin, the background shape was determined by fitting a 3\textsuperscript{rd} order polynomial at least six widths to either side of the mass. The parameters of this fit were then fixed and a Gaussian of fixed mean and width was added to the function. The only parameter allowed to vary at this point was the height of the Gaussian peak. After this new function was fitted to the data, the integral of Gaussian, two standard deviations to each side of the mean, was calculated as the number of events ``detected'' and the integral of the polynomial part over the same range gave the number of background events. The measured yield was verified to be consistent with zero and the two standard deviation error was taken as the upper limit of the measured yield shown in Fig.~\ref{fig:high.xi.yield}. The resulting flux-corrected yield is shown in Fig.~\ref{fig:high.xi.yieldflux}. The acceptance, using the model described in Sec.~\ref{sec:results.xi1320.accept} is shown in Fig.~\ref{fig:high.xi.accept}. Combining these, including the scale factor of 34\% yields the total cross section upper limits shown in Fig.~\ref{fig:high.xi.ulimits}. These are the results over several energy bins, each 250~MeV wide; Tables~\ref{tab:high.xi.yield}, \ref{tab:high.xi.accept} and \ref{tab:high.xi.ulimits} list the results equivalent results integrated over the whole energy range.





\begin{figure}[bhp]\centering
    \includegraphics[width=0.7\columnwidth]{\figures/mmkk/high_xi_ulimit_yield.eps}
    \caption[Higher Mass \texorpdfstring{$\Xi^*$}{Xi*} Measured Yield Upper Limits]{\label{fig:high.xi.yield}The upper limits of the measured yield of the possible $\Xi$ states at 1620, 1690 and 1820~MeV via the missing mass off $\Kp\Kp$ in the reaction $\gammaup \p \rightarrow \p \Kp \Kp X^{--}$. Shown are the data with selections described in Tables~\ref{tab:vtime.cuts} and \ref{tab:kpkp.cuts} including the \abbr{TOF} energy deposit cut.}
\end{figure}

\begin{figure}[bhpt]\centering
    \includegraphics[width=0.7\columnwidth]{\figures/mmkk/high_xi_ulimit_yieldfluxcorr.eps}
    \caption[Higher Mass \texorpdfstring{$\Xi^*$}{Xi*} Measured Yield Upper Limits]{\label{fig:high.xi.yieldflux}The upper limits of the flux-corrected measured yield of the possible $\Xi$ states at 1620, 1690 and 1820~MeV via the missing mass off $\Kp\Kp$ in the reaction $\gammaup \p \rightarrow \p \Kp \Kp X^{--}$. Shown are the data with selections described in Tables~\ref{tab:vtime.cuts} and \ref{tab:kpkp.cuts} including the \abbr{TOF} energy deposit cut.}
\end{figure}

\begin{figure}[bhpt]\centering
    \includegraphics[width=0.7\columnwidth]{\figures/mmkk/high_xi_accept.eps}
    \caption[Higher Mass \texorpdfstring{$\Xi*$}{Xi*} Acceptances]{\label{fig:high.xi.accept}The acceptances of the $\Xi^*$ states at 1620, 1690 and 1820~MeV.}
\end{figure}

\begin{figure}[bhpt]\begin{center}
    \includegraphics[width=\figwidth]{\figures/mmkk/high_xi_ulimit.eps}
    \caption[$\Xi^*$ Total Cross Section Upper Limits]{\label{fig:high.xi.ulimits}Total cross section upper limits for photoproduction of $\Xi^-$(1620), $\Xi^-$(1690) and $\Xi^-$(1820).}
\end{center}\end{figure}



\begin{table}[bh]\begin{center}
\caption[High Mass \texorpdfstring{$\Xi^-$}{Xi-} Yield Upper Limits]{\label{tab:high.xi.yield}Upper limits for the yield of the higher mass $\Xi$ states over the entire energy range (3.5--5.7~GeV) with a confidence level of 90\%.}
\begin{tabular}{ccc}
\hline \hline
 & $<N$ & $<Y$ (pb) \\
\hline
$\Xi^-$(1620) & 274 & 5.88 \\
$\Xi^-$(1690) & 290 & 7.36 \\
$\Xi^-$(1820) & 244 & 7.41 \\
\hline \hline
\end{tabular}
\end{center}\end{table}

\begin{table}[bhpt]\begin{center}
\caption[High Mass \texorpdfstring{$\Xi^-$}{Xi-} Acceptances]{\label{tab:high.xi.accept}Acceptances calculated over the entire energy range (3.5--5.7~GeV) for the higher mass $\Xi$ states. Shown are the number of events generated ($G$), reconstructed ($R$) and the acceptance ($A$) with an error given by Eq.~\ref{eqn:accept.err}.}
\begin{tabular}{ccccc}
\hline \hline
 & $G$ & $R$ & $A$ & $\delta A$ \\
 & ($10^6$) & ($10^3$) & (\%) & (\%) \\
\hline
$\Xi^-$(1620) & 2.077 & 23.6 & 1.14 & 0.01 \\
$\Xi^-$(1690) & 2.000 & 22.9 & 1.15 & 0.01 \\
$\Xi^-$(1820) & 1.942 & 19.9 & 1.03 & 0.01 \\
\hline \hline
\end{tabular}
\end{center}\end{table}

\begin{table}[bhpt]\begin{center}
\caption[High Mass \texorpdfstring{$\Xi^-$}{Xi-} Total Cross Section Upper Limits]{\label{tab:high.xi.ulimits}Upper limits for the total cross section of the higher mass $\Xi$ states over the entire energy range (3.5--5.7~GeV) with a confidence level of 90\% with and without the scale factor of 34\% applied.}
\begin{tabular}{ccc}
\hline \hline
 & no scale factor & 34\% scale factor \\
 & $<\sigma$ (pb) & $<\sigma$ (nb) \\
\hline
$\Xi^-$(1620) & 516 & 0.78 \\
$\Xi^-$(1690) & 640 & 0.97 \\
$\Xi^-$(1820) & 720 & 1.09 \\
\hline \hline
\end{tabular}
\end{center}\end{table}

\clearpage
